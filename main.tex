%%%%%%%%%%%%%%%%%%%%%%%%%%%%%%%%%%%%%%%%%
% Presentación Beamer - Plantilla LaTeX
% Versión 2.0 (8 de marzo de 2022)
% Plantilla original: https://www.LaTeXTemplates.com
% Autor: Vel (vel@latextemplates.com)
% Licencia: CC BY-NC-SA 4.0

% Este modelo de presentación fue 
% creado a partir del modelo de Giovanni Spadaro.
% Disponible en: https://github.com/Giovo17/presentation-template-unict-lm-data
%
% Adaptado por Lautaro Lombardi para crear una versión especial 
% para los estudiantes de Facultad de Matemática, Astronomía, Física y Computación (FAMAF - UNC).
% Disponible en: https://github.com/lautalom/beamer_FAMAF
%%%%%%%%%%%%%%%%%%%%%%%%%%%%%%%%%%%%%%%%%

%----------------------------------------------------------------------------------------
% CLASE DEL DOCUMENTO Y CONFIGURACIONES BÁSICAS
%----------------------------------------------------------------------------------------
\documentclass[
    11pt,               % Tamaño estándar de la fuente
    % t,                % Alinear verticalmente en la parte superior
    %aspectratio=169,   % Definir proporción 16:9
]{beamer}
\graphicspath{{img/}}         % Define el directorio de las imágenes

%---------------------------------------------------------------------------------------
% PAQUETES NECESARIOS
%----------------------------------------------------------------------------------------
\usepackage{
    booktabs,     % Mejora la apariencia de las líneas en tablas
    palatino,     % Define Palatino como fuente principal
    subcaption    % Soporte para subfiguras
}
\usepackage[spanish]{babel}  % Set babel to use Spanish language
\usepackage[default]{opensans}  % Define Open Sans como fuente secundaria
\input{config/code_langs}       % Importa configuraciones para resaltar código

%---------------------------------------------------------------------------------------
% CONFIGURACIÓN DEL TEMA
%----------------------------------------------------------------------------------------
% Tema Base
\usetheme{Boadilla}                          % Define el tema principal
\useinnertheme{circles}                      % Tema interno con círculos
\useoutertheme{miniframes}                   % Tema externo con miniframes
\setbeamertemplate{navigation symbols}{}     % Elimina símbolos de navegación

% Colores Personalizados
\definecolor{primaryColor}{RGB}{11, 139, 171} % Color primario - cian Famaf oficial
\definecolor{secondaryColor}{RGB}{0, 104, 128}    % Color secundario - azul verdoso

% Configuraciones de Colores
\setbeamercolor{structure}{fg=primaryColor}
\setbeamercolor{palette primary}{bg=primaryColor, fg=white}
\setbeamercolor{palette secondary}{bg=secondaryColor, fg=white}
\setbeamercolor{title}{bg=primaryColor, fg=white}

% Colores del Encabezado y Pie de Página
\setbeamercolor{headline}{bg=secondaryColor, fg=white}
\setbeamercolor{section in head/foot}{bg=primaryColor, fg=white}
\setbeamercolor{subsection in head/foot}{bg=secondaryColor, fg=white}
\setbeamercolor{author in head/foot}{bg=primaryColor, fg=white}
\setbeamercolor{title in head/foot}{bg=secondaryColor, fg=white}
\setbeamercolor{date in head/foot}{bg=primaryColor, fg=white}
\setbeamercolor{page number in head/foot}{bg=primaryColor, fg=white}

%----------------------------------------------------------------------------------------
% BIBLIOGRAFÍA
%----------------------------------------------------------------------------------------
\bibliographystyle{plain}
%----------------------------------------------------------------------------------------
% INFORMACIÓN DE LA PRESENTACIÓN
%----------------------------------------------------------------------------------------
\title[Swaptions en LMM]{Valoración de Swaptions en el Modelo de Mercado Libor}          % [Versión corta]{Versión completa}
\author[Lautaro Lombardi]{Lautaro Lombardi}        % [Versión corta]{Nombre completo}
\institute[FAMAF]{Facultad de Matemática, Astronomía, Física y Computación \\ (FAMAF - UNC) }
\date[2025]{Marzo 2025}                                % [Versión corta]{Fecha completa}   

%----------------------------------------------------------------------------------------
% INICIO DEL DOCUMENTO
%----------------------------------------------------------------------------------------
\begin{document}
% Diapositiva de título con logo
\newtheorem{defin}{Definición}[section]
\newtheorem{thm}{Teorema}[section]
\newtheorem{prop}{Proposicion}[section]
\newtheorem{obs}{Observación}[section]
\newtheorem{notat}{Notación}[section]

\begin{frame}
    \begin{figure}
        \includegraphics[width=0.50\linewidth]{img/famaf.png} % Adjusted width
    \end{figure}
    \titlepage
\end{frame}

% Sumario
\begin{frame}
    \frametitle{Estructura de la presentación}
    \tableofcontents
\end{frame}

% Inclusión de las secciones
\section{Tasas de interés y derivados}

\begin{frame}
% center the title
    \begin{center}
        \textbf{\huge Tasas de interés y derivados}
    \end{center}
\end{frame}
%----------------------------------------------------------------------------------------

\begin{frame}
    \frametitle{Tasa de interés simple}
    Si A le presta un monto \$1 en tiempo $T_1$ a B y B le paga un monto \$1 en tiempo $T_2$ más cierto interés K entre $T_1$ y $T_2$. 
    \begin{figure}[h]
       \centering
       \includegraphics[width=\textwidth]{img/cap1/accrual_simple.png}
       \caption{Tasa simple}
       \label{accrual_simple}
   \end{figure}
\end{frame}

%----------------------------------------------------------------------------------------

\begin{frame}
    \frametitle{Sistemas de Capitalización}
    \begin{itemize}
        \item \textbf{Capitalización Simple:} El interés se calcula sobre el capital inicial. Un interés de \$100 a una tasa de 10\% anual con capitalización semestral durante 3 años resulta en $100 \times (1+\frac{0.1}{2} \times 6) = 130$.
        \item \textbf{Capitalización Compuesta Discreta:} El interés se calcula sobre el capital inicial más los intereses acumulados. Invertir \$100 a una tasa de 10\% anual durante 3 años con capitalización semestral resulta en $100 \times (1+0.1/2)^{2 \times 3} = 133.1$.  
        \item \textbf{Capitalización Compuesta Continua:} El interés se calcula sobre el capital inicial más los intereses acumulados, pero en un límite continuo. Invertir \$100 a una tasa de 10\% anual durante 3 años con capitalización continua resulta en $100 \times e^{0.1 \times 3} = 134.98$. 
    \end{itemize}

\end{frame}

%----------------------------------------------------------------------------------------

\begin{frame}
    \frametitle{Bonos cupon cero}
    Componentes:
    \begin{itemize}
        \item \textbf{Valor nominal (N):} Monto a pagar al vencimiento.
        \item \textbf{Tasa de interés (r):} Tasa \textit{cero} anual.
        \item \textbf{Tiempo hasta el vencimiento (T):} Tiempo en años hasta el vencimiento.
        \item \textbf{Precio (P):} Precio actual del bono.
    \end{itemize}
    % side by side equations
    \begin{columns}
        \column{0.4\textwidth}
            \begin{equation*}
                P = \frac{N}{(1+r)^T}
            \end{equation*}
        \column{0.4\textwidth}
            \begin{equation*}
                P = N \cdot e^{-rT}
            \end{equation*}
    \end{columns}
\end{frame}

%----------------------------------------------------------------------------------------

\begin{frame}
    \frametitle{Valoración de un bono con cupones}
    Consideremos un bono de principal \$100, madurez en 2 años y tasa cupón 5\% semianual.\\
    Supongamos que las tasas cero a 6 meses, 1 año, 1.5 años y 2 años son 4\%, 4.5\%, 5\% y 5.5\% respectivamente.\\
    \begin{itemize}
        \item \textbf{Flujos de caja:} \$2.5, \$2.5, \$2.5, \$102.5
        \item \textbf{Tasas cero:} 4\%, 4.5\%, 5\%, 5.5\%
        \item \textbf{Precios:} $\$2.5 \cdot e^{-0.04 \cdot 0.5} + \$2.5 \cdot e^{-0.045 \cdot 1} + \$2.5 \cdot e^{-0.05 \cdot 1.5} + \$102.5 \cdot e^{-0.055 \cdot 2}$
        \item \textbf{Precio total:} 98.98
    \end{itemize}
\end{frame}
%----------------------------------------------------------------------------------------

\begin{frame}
    \begin{figure}[h]
       \centering
       \includegraphics[width=\textwidth]{img/cap1/bono_cupon.jpg}
       \caption{Valoración de un bono con cupones}
       \label{bono_cupon}
   \end{figure}
\end{frame}
%----------------------------------------------------------------------------------------

\begin{frame}
    \frametitle{Modelo de tasas de interés}
    \begin{defin}[cuenta de moneda]
        Activo que capitaliza intereses a una tasa de interés r libre de riesgo.\\
        \begin{equation*}
            B(0) = 1 
        \end{equation*}
        \begin{equation*}
            dB(t) = rB(t)dt
        \end{equation*}
        \begin{equation*}
            B(t) = B(0)e^{rt}
        \end{equation*}
        \begin{equation*}
            B(t) = B(0)e^{\int_0^t r(s)ds}
        \end{equation*}
    \end{defin}
\end{frame}

%----------------------------------------------------------------------------------------

\begin{frame}
    \frametitle{Valor inicial de un flujo de caja}
    \begin{defin}[Valor inicial]
        El valor inicial de \$1 en el tiempo T es A tal que:
        \begin{equation*}
            A \times B(T) = 1
        \end{equation*}
        \begin{equation*}
            A = \frac{1}{B(T)} = e^{-rT}
        \end{equation*}
        \begin{equation*}
            A = e^{-\int_0^T r(s)ds}
        \end{equation*}
    \end{defin}
\end{frame}
%----------------------------------------------------------------------------------------

\begin{frame}
    \frametitle{Valor presente}
    \begin{defin}[Factor de descuento]
        En $t<T$ el valor presente de un flujo de caja $1$ en T es:
        \begin{equation*}
            A \times B(t) = \frac{B(t)}{B(T)}
        \end{equation*}
        \begin{equation*}
           D_B(t,T) = \frac{B(t)}{B(T)}
        \end{equation*}
    \end{defin}
\end{frame}

\begin{frame}
    \frametitle{Bono cupón cero}
    \begin{defin}[Bono cupón cero]
        Un bono cupón cero es un activo que paga \$1 en el tiempo T.\\
        \begin{equation*}
            P(T, T) = 1  
        \end{equation*}
    \end{defin}
\end{frame}

%----------------------------------------------------------------------------------------
\section{Movimiento Browniano}

%----------------------------------------------------------------------------------------

\begin{frame}
%center the title
    \begin{center}
        \textbf{\huge Movimiento Browniano}
    \end{center}
\end{frame}

%----------------------------------------------------------------------------------------

\begin{frame}
    \frametitle{Procesos Estocásticos}
    \begin{itemize}
        \item Variables que cambian de valor en el tiempo de forma incierta.
        \item Clasificación: tiempo discreto o continuo, variable continua o discreta.
        \item Definición formal: un proceso estocástico es una colección de variables aleatorias indexadas por el tiempo.
        \[X: \Omega \times I \rightarrow \mathbb{R}\]
    \end{itemize}
\end{frame}

%----------------------------------------------------------------------------------------

\begin{frame}
    \frametitle{Definición del Proceso \( S_t(\omega_1 \omega_2 \omega_3) \)}
    
    \[
    S_0(\omega_1 \omega_2 \omega_3) = 4 \quad \text{para todo } \omega_1 \omega_2 \omega_3 \in \Omega_3,
    \]
    
    \[
    S_1(\omega_1 \omega_2 \omega_3) =
    \begin{cases}
        8 & \text{si } \omega_1 = C, \\
        2 & \text{si } \omega_1 = N,
    \end{cases}
    \]
    
    \[
    S_2(\omega_1 \omega_2 \omega_3) =
    \begin{cases}
        16 & \text{si } \omega_1 = \omega_2 = C, \\
        4 & \text{si } \omega_1 \ne \omega_2, \\
        1 & \text{si } \omega_1 = \omega_2 = N,
    \end{cases}
    \]
    
    \[
    S_3(\omega_1 \omega_2 \omega_3) =
    \begin{cases}
        32 & \text{si } \omega_1 = \omega_2 = \omega_3 = C, \\
        8 & \text{si hay dos caras y una cruz}, \\
        2 & \text{si hay una cara y dos cruces}, \\
        0.5 & \text{si } \omega_1 = \omega_2 = \omega_3 = N.
    \end{cases}
    \]

\end{frame}

%----------------------------------------------------------------------------------------

\begin{frame}
    \frametitle{Evolución de $S_t$}
    \begin{figure}
        \centering
        \includegraphics[width=0.8\textwidth]{img/cap2/S_t.jpg}
        \caption{Evolución del proceso $S_t$ en el tiempo.}
        \label{fig:evolucion}
    \end{figure}
\end{frame}

%----------------------------------------------------------------------------------------
\begin{frame}
    \frametitle{Movimiento Browniano}
    \begin{itemize}
        \item Proceso continuo que comienza en 0: \(W(0) = 0\).
        \item Incrementos independientes y distribuidos normalmente.
        \item Propiedades:
        \begin{itemize}
            \item \(\mathbb{E}[W(t)] = 0\).
            \item \(\text{Var}[W(t)] = t\).
        \end{itemize}
    \end{itemize}
\end{frame}

%----------------------------------------------------------------------------------------

\begin{frame}
    \frametitle{Valor Esperado Condicional}
    \begin{itemize}
        \item Definición: \(\mathbb{E}[X|\mathcal{G}]\) satisface medibilidad y promedio parcial.
        \item Propiedades:
        \begin{itemize}
            \item Linealidad.
            \item Condicionamiento iterado.
            \item Desigualdad de Jensen.
        \end{itemize}
    \end{itemize}
\end{frame}

%----------------------------------------------------------------------------------------

\begin{frame}
    \frametitle{Martingalas}
    \begin{itemize}
        \item Proceso estocástico sin deriva.
        \item Propiedad clave: \(\mathbb{E}[M(t)|\mathcal{F}(s)] = M(s)\).
        \item Ejemplo: Movimiento Browniano es una martingala.
    \end{itemize}
\end{frame}

%----------------------------------------------------------------------------------------

\begin{frame}
    \frametitle{Proceso de Itô}
    \begin{itemize}
        \item Modelo para capturar la naturaleza estocástica de los precios de activos financieros.
        \item Representación general:
        \[dX = \mu(X, t) \, dt + \sigma(X, t) \, dW\]
        \item \(\mu\): tasa de deriva. \(\sigma\): volatilidad.
    \end{itemize}
\end{frame}

%----------------------------------------------------------------------------------------

\begin{frame}
    \frametitle{Lema de Itô}
    \begin{itemize}
        \item Relaciona una función \(G(X, t)\) con el proceso de Itô subyacente.
        \item Fórmula:
        \[dG = \frac{\partial G}{\partial t} \, dt + \frac{\partial G}{\partial X} \, dX + \frac{1}{2} \frac{\partial^2 G}{\partial X^2} \, (\sigma^2 \, dt)\]
        \item Aplicación: valuación de derivados financieros.
    \end{itemize}
\end{frame}
% Diapositiva final
\begin{frame}
    \begin{center}
        {\Huge ¡Fin de la presentación!}
    \end{center}
\end{frame}

\end{document}

